% Options for packages loaded elsewhere
% Options for packages loaded elsewhere
\PassOptionsToPackage{unicode}{hyperref}
\PassOptionsToPackage{hyphens}{url}
\PassOptionsToPackage{dvipsnames,svgnames,x11names}{xcolor}
%
\documentclass[
  letterpaper,
  DIV=11,
  numbers=noendperiod]{scrreprt}
\usepackage{xcolor}
\usepackage{amsmath,amssymb}
\setcounter{secnumdepth}{5}
\usepackage{iftex}
\ifPDFTeX
  \usepackage[T1]{fontenc}
  \usepackage[utf8]{inputenc}
  \usepackage{textcomp} % provide euro and other symbols
\else % if luatex or xetex
  \usepackage{unicode-math} % this also loads fontspec
  \defaultfontfeatures{Scale=MatchLowercase}
  \defaultfontfeatures[\rmfamily]{Ligatures=TeX,Scale=1}
\fi
\usepackage{lmodern}
\ifPDFTeX\else
  % xetex/luatex font selection
\fi
% Use upquote if available, for straight quotes in verbatim environments
\IfFileExists{upquote.sty}{\usepackage{upquote}}{}
\IfFileExists{microtype.sty}{% use microtype if available
  \usepackage[]{microtype}
  \UseMicrotypeSet[protrusion]{basicmath} % disable protrusion for tt fonts
}{}
\makeatletter
\@ifundefined{KOMAClassName}{% if non-KOMA class
  \IfFileExists{parskip.sty}{%
    \usepackage{parskip}
  }{% else
    \setlength{\parindent}{0pt}
    \setlength{\parskip}{6pt plus 2pt minus 1pt}}
}{% if KOMA class
  \KOMAoptions{parskip=half}}
\makeatother
% Make \paragraph and \subparagraph free-standing
\makeatletter
\ifx\paragraph\undefined\else
  \let\oldparagraph\paragraph
  \renewcommand{\paragraph}{
    \@ifstar
      \xxxParagraphStar
      \xxxParagraphNoStar
  }
  \newcommand{\xxxParagraphStar}[1]{\oldparagraph*{#1}\mbox{}}
  \newcommand{\xxxParagraphNoStar}[1]{\oldparagraph{#1}\mbox{}}
\fi
\ifx\subparagraph\undefined\else
  \let\oldsubparagraph\subparagraph
  \renewcommand{\subparagraph}{
    \@ifstar
      \xxxSubParagraphStar
      \xxxSubParagraphNoStar
  }
  \newcommand{\xxxSubParagraphStar}[1]{\oldsubparagraph*{#1}\mbox{}}
  \newcommand{\xxxSubParagraphNoStar}[1]{\oldsubparagraph{#1}\mbox{}}
\fi
\makeatother


\usepackage{longtable,booktabs,array}
\usepackage{calc} % for calculating minipage widths
% Correct order of tables after \paragraph or \subparagraph
\usepackage{etoolbox}
\makeatletter
\patchcmd\longtable{\par}{\if@noskipsec\mbox{}\fi\par}{}{}
\makeatother
% Allow footnotes in longtable head/foot
\IfFileExists{footnotehyper.sty}{\usepackage{footnotehyper}}{\usepackage{footnote}}
\makesavenoteenv{longtable}
\usepackage{graphicx}
\makeatletter
\newsavebox\pandoc@box
\newcommand*\pandocbounded[1]{% scales image to fit in text height/width
  \sbox\pandoc@box{#1}%
  \Gscale@div\@tempa{\textheight}{\dimexpr\ht\pandoc@box+\dp\pandoc@box\relax}%
  \Gscale@div\@tempb{\linewidth}{\wd\pandoc@box}%
  \ifdim\@tempb\p@<\@tempa\p@\let\@tempa\@tempb\fi% select the smaller of both
  \ifdim\@tempa\p@<\p@\scalebox{\@tempa}{\usebox\pandoc@box}%
  \else\usebox{\pandoc@box}%
  \fi%
}
% Set default figure placement to htbp
\def\fps@figure{htbp}
\makeatother





\setlength{\emergencystretch}{3em} % prevent overfull lines

\providecommand{\tightlist}{%
  \setlength{\itemsep}{0pt}\setlength{\parskip}{0pt}}



 


\KOMAoption{captions}{tableheading}
\makeatletter
\@ifpackageloaded{bookmark}{}{\usepackage{bookmark}}
\makeatother
\makeatletter
\@ifpackageloaded{caption}{}{\usepackage{caption}}
\AtBeginDocument{%
\ifdefined\contentsname
  \renewcommand*\contentsname{Table of contents}
\else
  \newcommand\contentsname{Table of contents}
\fi
\ifdefined\listfigurename
  \renewcommand*\listfigurename{List of Figures}
\else
  \newcommand\listfigurename{List of Figures}
\fi
\ifdefined\listtablename
  \renewcommand*\listtablename{List of Tables}
\else
  \newcommand\listtablename{List of Tables}
\fi
\ifdefined\figurename
  \renewcommand*\figurename{Figure}
\else
  \newcommand\figurename{Figure}
\fi
\ifdefined\tablename
  \renewcommand*\tablename{Table}
\else
  \newcommand\tablename{Table}
\fi
}
\@ifpackageloaded{float}{}{\usepackage{float}}
\floatstyle{ruled}
\@ifundefined{c@chapter}{\newfloat{codelisting}{h}{lop}}{\newfloat{codelisting}{h}{lop}[chapter]}
\floatname{codelisting}{Listing}
\newcommand*\listoflistings{\listof{codelisting}{List of Listings}}
\makeatother
\makeatletter
\makeatother
\makeatletter
\@ifpackageloaded{caption}{}{\usepackage{caption}}
\@ifpackageloaded{subcaption}{}{\usepackage{subcaption}}
\makeatother
\usepackage{bookmark}
\IfFileExists{xurl.sty}{\usepackage{xurl}}{} % add URL line breaks if available
\urlstyle{same}
\hypersetup{
  pdftitle={Rafa Abdussalam D},
  pdfauthor={13523133 Rafa Abdussalam D},
  colorlinks=true,
  linkcolor={blue},
  filecolor={Maroon},
  citecolor={Blue},
  urlcolor={Blue},
  pdfcreator={LaTeX via pandoc}}


\title{Rafa Abdussalam D}
\usepackage{etoolbox}
\makeatletter
\providecommand{\subtitle}[1]{% add subtitle to \maketitle
  \apptocmd{\@title}{\par {\large #1 \par}}{}{}
}
\makeatother
\subtitle{Portfolio Asesmen UTS-1 Komunikasi Interpersonal}
\author{13523133 Rafa Abdussalam D}
\date{2027-07-10}
\begin{document}
\maketitle

\renewcommand*\contentsname{Table of contents}
{
\hypersetup{linkcolor=}
\setcounter{tocdepth}{2}
\tableofcontents
}

\bookmarksetup{startatroot}

\chapter*{Selamat Berjumpa}\label{selamat-berjumpa}
\addcontentsline{toc}{chapter}{Selamat Berjumpa}

\markboth{Selamat Berjumpa}{Selamat Berjumpa}

Perkenalkan Saya Rafa Abdussalam Danadyaksa, biasa dipanggil Rafa. Saya
adalah seorang mahasiswa jurusan Teknik Informatika di Institut
Teknologi Bandung (ITB). Saya memiliki minat besar pada dunia teknologi,
pengembangan sistem, dan pembelajaran manusia. Hobi saya bermacam-macam,
tergantung mood. Kadang suka bermain alat musik gitar, olahraga lari dan
renang, atau sekadar menonton film dan mendengarkan musik. Saya juga
senang bereksperimen dengan coding, mulai mengekplorasi berbagai bidang
yang ada di dunia teknologi informasi.

\bookmarksetup{startatroot}

\chapter{UTS-1 All About Me}\label{uts-1-all-about-me}

\begin{figure}[H]

{\centering \includegraphics[width=1.5625in,height=2.08333in]{All_About_me/foto-Rafa.png}

}

\caption{About Me}

\end{figure}%

Selamat datang \#\# \textbf{Haloo! } Perkenalkan, nama saya \textbf{Rafa
Abdussalam Danadyaksa} biasa dipanggil \textbf{Rafa}. Saya adalah
seorang mahasiswa jursan \textbf{Teknik Informatika} yang memiliki minat
besar pada dunia \textbf{teknologi, pengembangan sistem, dan
pembelajaran manusia}. Bagi saya, belajar bukan sekadar mengejar nilai,
tetapi sebuah perjalanan untuk mengenal diri sendiri --- tentang
bagaimana saya berpikir, bekerja, gagal, dan bangkit lagi. Setiap baris
kode, setiap kesalahan, dan setiap proses refleksi telah membantu saya
memahami siapa saya dan apa nilai yang ingin saya bawa ke dunia ini.

\subsection{\texorpdfstring{\textbf{1. Seni Memberi Makna: Kekuatan
Super Anda dalam
Bernalar}}{1. Seni Memberi Makna: Kekuatan Super Anda dalam Bernalar}}\label{seni-memberi-makna-kekuatan-super-anda-dalam-bernalar}

Saya percaya bahwa setiap peristiwa dalam hidup baik yang menyenangkan
maupun menyakitkan --- membawa pesan tersembunyi. Kemampuan saya untuk
\textbf{melihat pola, memahami sebab-akibat, dan menghubungkannya dengan
nilai-nilai diri} adalah kekuatan utama saya dalam memberi makna. Inilah
bentuk \emph{penalaran otobiografis} saya: saya tidak hanya mengingat
apa yang terjadi, tapi juga \textbf{mengapa itu penting} bagi perjalanan
saya.

\subsection{\texorpdfstring{\textbf{2. Menghadapi Tantangan dengan Rasa
Ingin
Tahu}}{2. Menghadapi Tantangan dengan Rasa Ingin Tahu}}\label{menghadapi-tantangan-dengan-rasa-ingin-tahu}

Aku yakin kalau rasa mau tahu merupakan bahan bakar utama perkembangan.
Aku senantiasa tertarik untuk berupaya hal- hal baru: mulai dari membuat
web simpel dengan HTML serta PHP, sampai belajar tentang informasi
science serta kecerdasan buatan.

Tantangan terbesar tidaklah kompleksitas teknologinya,
namun\textbf{mengatur mindset supaya tidak gampang menyerah}. Aku
belajar kalau tidak terdapat kegagalan yang percuma--- tiap kesalahan
merupakan bagian dari proses mengarah versi diri yang lebih matang. ---

\subsection{\texorpdfstring{\textbf{3. Mimpi dan Arah ke
Depan}}{3. Mimpi dan Arah ke Depan}}\label{mimpi-dan-arah-ke-depan}

Mimpi ku Ke depannya, aku ingin terus tumbuh menjadi seorang yang tidak
hanya memahami teknologi, tetapi juga\textbf{menggunakannya untuk
menghasilkan dampak sosial yang nyata}. Aku mau berkontribusi dalam
proyek- proyek yang membantu masyarakak seperti melalui aplikasi
pembelajaran, sistem data publik, ataupun platform yang menunjang
kreativitas anak muda Indonesia.

Aku mau petualangan akademik aku menjadi pijakan untuk suatu yang lebih
besar:\textbf{menjadi manusia yang sanggup berpikir kritis, berempati,
serta memberi arti melalui karya.}

\begin{center}\rule{0.5\linewidth}{0.5pt}\end{center}

\bookmarksetup{startatroot}

\chapter{UTS-2 My Songs for You}\label{uts-2-my-songs-for-you}

Lagu ``Monokrom'' merupakan ungkapan terima kasih kepada orang- orang
yang sempat muncul serta berikan warna dalam hidup baik keluarga, teman,
orang terdekat, maupun masalalu yang telah membentuk diri kita.

Meski kata monokrom berarti ``satu warna'' (hitam- putih), lagu ini
justru menggambarkan bagaimana kenangan, walaupun bisa jadi tidak
senantiasa indah, tetap memiliki nilai serta keindahan tersendiri.

\url{https://www.youtube.com/watch?v=QqJ-Vp8mvbk}, with lyrics

{[}Lirik ``Monokrom''{]}

{[}Verse 1{]} Lembaran foto hitam putih Aku coba ingat lagi warna bajumu
kala itu Kali pertama di hidupku Manusia lain memelukku

{[}Verse 2{]} Lembaran foto hitam putih Aku coba ingat lagi wangi rumah
di sore itu Kue cokelat balon warna-warni Pesta hari ulang tahunku

{[}Chorus{]} Dimanapun kalian berada Kukirimkan terima kasih Untuk warna
dalam hidupku dan banyak kenangan indah Kau melukis aku

{[}Verse 3{]} Lembaran foto hitam putih Kembali teringat malam
kuhitung-hitung bintang Saat mataku sulit tidur Suaramu buatku lelap You
might also like {[}Chorus{]} Dimanapun kalian berada Kukirimkan terima
kasih Untuk warna dalam hidupku dan banyak kenangan indah Kau melukis
aku

{[}Bridge{]} Kita tak pernah tahu berapa lama kita Diberi waktu Jika aku
pergi lebih dulu jangan lupakan aku Ini lagu untukmu Ungkapan terima
kasihku

{[}Outro{]} Lembar monokrom hitam putih Aku coba ingat warna demi warna
di hidupku Tak akan 'ku mengenal cinta Bila bukan karena hati baikmu

\bookmarksetup{startatroot}

\chapter{UTS- 3 My Stories for You}\label{uts--3-my-stories-for-you}

\subsubsection{Bagian 1: Rasa Khawatir( serta Dinding yang Kubangun
Sendiri)}\label{bagian-1-rasa-khawatir-serta-dinding-yang-kubangun-sendiri}

Sedari SMA, terdapat satu nama yang tidak berani aku sebut sebagai
tujuan: ITB. Di kepala aku, tempat itu merupakan tingkat lain, diisi
oleh orang- orang yang levelnya'' tidak terjangkau''. Aku membangun
sebuah narasi, suatu dinding pertahanan guna melindungi ego aku. Aku
meyakinkan diri sendiri bahwa aku tidak pantas, dan bahkan, aku
tidak\emph{ingin} ke sana. Tiap kali terdapat teman ataupun saudara
bertanya,'' Nanti ingin kuliah di mana?''

Aku sudah siapkan jawaban mantap.'' UI,'' kata aku.'' Aku tidak ingin di
ITB.''

Serta jika mereka bertanya mengapa, aku hendak meningkatkan dengan
percaya,'' ITB berat.''

Pasti saja,'' berat'' merupakan kata lain untuk'' saya takut gagal''.

Aku lebih baik berpura- pura tidak ingin, daripada berupaya serta
meyakinkan kalau ketakutan aku benar.

\subsubsection{Bagian 2: Kegagalan( Pelajaran dari Pintu yang
Tertutup)}\label{bagian-2-kegagalan-pelajaran-dari-pintu-yang-tertutup}

Sehingga, aku pun menempuh rencana aku. Aku mendaftar ke bermacam tes
masuk, menjauhi satu nama yang aku takuti itu. Saya mengerahkan seluruh
usaha saya ke pilihan- pilihan yang saya anggap'' nyaman'' serta''
cocok'' dengan kemampuan saya. Satu per satu, hasilnya keluar.

'' Kamu tidak diterima.''

Lagi.'' Kamu tidak diterima.''

Beberapa kali kegagalan aku terima. Penolakan demi penolakan. Pintu-
pintu yang aku yakini akan terbuka, malah tertutup rapat. Seluruh
rencana saya hancur. Dinding yang aku bangun untuk melindungi ego aku
dari kegagalan ITB, saat ini lagi dihantam oleh kegagalan dari tempat
lain.

\subsubsection{Bagian 3: Dorongan ``Coba Saja'' (Lompatan Tanpa
Keyakinan)}\label{bagian-3-dorongan-coba-saja-lompatan-tanpa-keyakinan}

Di tengah keputusasaan itu, keluarga saya tiba dengan masukan yang
paling aku hindarkan.

'' Coba aja tes mandiri ITB.''

Aku menolak. Tetapi dorongan itu tiba lagi, kali ini ``agak sedikit
maksa''.

'' Coba aja,'' kata mereka.'' Siapa tau dapat.''

Kalimat ``siapa tau dapat'' itu terdengar berbeda di telinga saya yang
sudah lelah ditolak.

Itu bukan kalimat yang menuntut kesuksesan, itu hanya ajakan untuk
berupaya. Tidak terdapat ekspektasi. Tidak terdapat beban. Dengan sisa-
sisa keberanian atau bisa jadi lebih tepatnya, kepasrahan saya
mendaftar.

Aku ``memberanikan diri'', bukan sebab aku seketika yakin diri, tetapi
sebab aku telah tidak memiliki apa- apa lagi buat dipertaruhkan.

\subsubsection{Bagian 4: Hasil (Jalan yang Tidak Pernah Saya
Pilih)}\label{bagian-4-hasil-jalan-yang-tidak-pernah-saya-pilih}

Hari pengumuman datang. Aku membukanya tanpa harapan, telah siap buat
memandang penolakan yang kesekian kali. Aku cuma mau ini semua segera
berakhir. Saya membaca layarnya.

\ldots{} Dan, dapat.

Aku diterima. Di salah satunya tempat yang aku bilang'' tidak ingin''.
Di salah satunya tempat yang aku anggap'' sangat berat''. Di salah
satunya tempat yang aku rasa'' tidak pantas'' buat aku. Hidup memiliki
cara yang lucu untuk menampilkan kalau kita salah. Jalur yang sangat aku
takuti, yang sangat aku jauhi, malah merupakan jalur yang terbuka kala
seluruh jalur lain tertutup. Aku belajar kalau rintangan terbesar aku
tidaklah beratnya tes ataupun hebatnya nama ITB. Rintangan terbesar aku
merupakan dinding'' tidak pantas'' yang aku bangun di kepala aku
sendiri.

\bookmarksetup{startatroot}

\chapter{UTS-4 My SHAPE}\label{uts-4-my-shape}

\begin{quote}
\textbf{Tujuan:} Merangkum rancangan diri (charter) agar saya melayani,
berkarya, dan memimpin secara paling selaras dengan karunia dan
pengalaman hidup saya. Dapat langsung ditempel ke halaman \textbf{UTS-4
--- My SHAPE}.
\end{quote}

\textbf{Peta SHAPE (singkat):}

\begin{itemize}
\tightlist
\item
  \textbf{S --- Strengths:} Kindness, Empathy, Problem Solving,
  Adaptability, Strategic Thinking.\\
\item
  \textbf{H --- Heart:} pengembangan teknologi bermanfaat, pembelajaran
  berbasis nilai, pertumbuhan pribadi \& komunitas.\\
\item
  \textbf{A --- Abilities (Kemampuan):} Pemrograman (Python/Web/Mobile),
  komunikasi, problem solving.\\
\item
  \textbf{P --- Personality (Gaya Kepribadian Kerja):} strategis \&
  reflektif, berorientasi visi \& nilai, analitis, kolaboratif.\\
\item
  \textbf{E --- Experiences (Pengalaman Kunci):} Kegagalan, Proyek
  Teknis, Kolaborasi Tim.
\end{itemize}

\begin{center}\rule{0.5\linewidth}{0.5pt}\end{center}

\section{1) S --- Strengths (Karunia)}\label{s-strengths-karunia}

Aku mempunyai kekuatan utama pada\textbf{empati},\textbf{ketekunan},
dan\textbf{berpikir strategis}. Aku cenderung memahami suasana orang
lain dikala saat sebelum berikan pemecahan, dan berupaya melindungi
ikatan baik dalam kerja regu. Dalam mengalami permasalahan, aku akan
terus berupaya sampai memperoleh pola maupun sistem yang bekerja. Aku
juga berpikir dalam kerangka jangka panjang; tidak hanya menuntaskan
tugas, namun membangun dasar supaya prosesnya bisa berulang dan
berkembang.

\begin{center}\rule{0.5\linewidth}{0.5pt}\end{center}

\section{2) H --- Heart (Nilai, Minat,
Kepedulian)}\label{h-heart-nilai-minat-kepedulian}

Saya terdorong oleh keinginan untuk \textbf{membuat sesuatu yang
bermanfaat dan bermakna}, bukan sekadar berfungsi.\\
Saya menikmati proses mengubah ide menjadi sistem nyata --- terutama
yang bisa membantu orang lain belajar, bekerja lebih mudah, atau merasa
terbantu.\\
Saya juga peduli pada \textbf{pertumbuhan pribadi} dan \textbf{hubungan
kolaboratif}, karena saya percaya keberhasilan sejati datang dari saling
dukung, bukan bersaing.

Hal-hal yang saya cintai meliputi: * Membangun sistem berbasis teknologi
yang membantu banyak orang. * Melihat tim berkembang dan saling
mempercayai. * Menghadirkan nilai kemanusiaan di tengah dunia digital.

\begin{center}\rule{0.5\linewidth}{0.5pt}\end{center}

\section{3) A --- Abilities (Kemampuan
Andal)}\label{a-abilities-kemampuan-andal}

Saya memiliki kemampuan andal yang berfokus pada pengembangan sistem dan
kolaborasi tim, yang terbagi menjadi:

\begin{itemize}
\tightlist
\item
  \textbf{Kemampuan Teknis (Hard Skills):}

  \begin{itemize}
  \tightlist
  \item
    \textbf{Pemrograman:} Menguasai Python, serta memiliki pengalaman
    dalam Web Development dan Mobile App Development.
  \end{itemize}
\item
  \textbf{Kemampuan Interpersonal (Soft Skills):}

  \begin{itemize}
  \tightlist
  \item
    \textbf{Komunikasi:} Mampu mempresentasikan ide dan membimbing orang
    lain.
  \item
    \textbf{Problem Solving:} Andal dalam menganalisis masalah dan
    merancang solusi yang sistematis.
  \item
    \textbf{Kolaborasi:} Mampu bekerja dalam tim (team work) dan adaptif
    terhadap perubahan.
  \end{itemize}
\end{itemize}

\begin{center}\rule{0.5\linewidth}{0.5pt}\end{center}

\section{4) P --- Personality (Gaya Kerja \&
Kolaborasi)}\label{p-personality-gaya-kerja-kolaborasi}

Gaya kerja dan kepribadian saya berfokus pada strategi, nilai, dan
kolaborasi. Saya cenderung menjadi seorang pemikir
\textbf{strategis-sistemik}, yang melihat gambaran besar sambil
memetakan bagian-bagian untuk mencapai visi. Saya juga seorang yang
\textbf{analitis}, menikmati proses dekonstruksi masalah untuk memahami
inti persoalan. Secara alami, saya \textbf{kolaboratif}, suka membangun
jejaring dan memberi ruang tumbuh bagi orang lain, namun juga
\textbf{tenang-tangguh}, mampu tetap fokus pada hasil jangka panjang
bahkan di bawah tekanan.

\begin{center}\rule{0.5\linewidth}{0.5pt}\end{center}

\section{5) E --- Experiences (Pengalaman
Pembentuk)}\label{e-experiences-pengalaman-pembentuk}

Saya dibentuk oleh berbagai pengalaman kunci, namun pelajaran paling
penting justru datang dari serangkaian \textbf{kegagalan} dan tantangan.
Pengalaman pembentuk utama saya bukanlah kesuksesan, melainkan proses
belajar dari berbagai kesalahan dan kegagalan, kegagalan akademik, atau
dinamika hidup. Justru kegagalan demi kegagalan inilah yang pada
akhirnya membentuk ketahanan, perspektif baru, dan kerendahan hati.
Pengalaman lain di bidang akademik \& proyek (membuat web marketplace,
review buku), kolaborasi, dan refleksi, namun semua itu diperkaya oleh
pelajaran yang didapat dari kegagalan.

\begin{center}\rule{0.5\linewidth}{0.5pt}\end{center}

\section{6) Tabel - Piagam Diri ``My
SHAPE''}\label{tabel---piagam-diri-my-shape}

\begin{longtable}[]{@{}
  >{\raggedright\arraybackslash}p{(\linewidth - 4\tabcolsep) * \real{0.2055}}
  >{\raggedright\arraybackslash}p{(\linewidth - 4\tabcolsep) * \real{0.3562}}
  >{\raggedright\arraybackslash}p{(\linewidth - 4\tabcolsep) * \real{0.4384}}@{}}
\toprule\noalign{}
\begin{minipage}[b]{\linewidth}\raggedright
\textbf{Komponen}
\end{minipage} & \begin{minipage}[b]{\linewidth}\raggedright
\textbf{Deskripsi Reflektif}
\end{minipage} & \begin{minipage}[b]{\linewidth}\raggedright
\textbf{Contoh Implementasi Nyata}
\end{minipage} \\
\midrule\noalign{}
\endhead
\bottomrule\noalign{}
\endlastfoot
\textbf{S -- Strengths} & Memiliki empati tinggi, ketekunan dalam
menyelesaikan masalah, dan kemampuan berpikir strategis yang membantu
melihat gambaran besar sekaligus detail teknis. & Membimbing teman
memahami proyek web; membuat struktur sistem agar tim mudah bekerja;
memperbaiki alur GitHub agar efisien. \\
\textbf{H -- Heart} & Dorongan utama saya adalah menghadirkan manfaat
melalui teknologi, membantu orang lain tumbuh, dan menciptakan sistem
yang mempermudah hidup atau pembelajaran. & Mengembangkan sistem web
yang bermanfaat (book review, pembelajaran digital). \\
\textbf{A -- Abilities (Skills)} & Menguasai pemrograman (Python, Web,
Mobile), memiliki kemampuan komunikasi dan problem solving yang kuat,
serta mampu beradaptasi dengan cepat terhadap perubahan. & Membuat
website dari nol, mengelola database, melakukan debugging dan optimasi
sistem. \\
\textbf{P -- Personality} & Analitis dan kolaboratif. & berkerja dengan
tim dan memberikan lingkungan yang kolaboratif \\
\textbf{E -- Experiences} & Pengalaman kegagalan dan tantangan membentuk
ketahanan, kedewasaan berpikir, serta kemampuan untuk belajar dari
kesalahan. & Belajar dari kegagalan seperti yang dijelaskan pada bagian
3 \\
\end{longtable}

\bookmarksetup{startatroot}

\chapter{UTS-5 My Personal Reviews}\label{uts-5-my-personal-reviews}

Berikut cara saya melakukan review: mengguan chatGPT, saya mengattach
\href{skor_uts.pdf}{file promt ChatGPT}, disertai perintah :``self
assess uts-1 sanpai uts-5 dari URL
`https://ii-2100.github.io/all-about-me/'\,''

ChatGPT melakukan self-assessment UTS-1 s.d. UTS-5 langsung dari laman
yang Anda berikan dan menilai memakai rubrik tugas UTS (skala 1--5 per
kriteria). Rekap skor siap diunduh sebagai CSV:
\href{sandbox:/mnt/data/UTS_self_assessment.csv}{Download CSV
ringkasan}.

\bookmarksetup{startatroot}

\chapter{Hasil Self-Assessment UTS (URL:
ii-2100.github.io/all-about-me)}\label{hasil-self-assessment-uts-url-ii-2100.github.ioall-about-me}

\section{Identifikasi}\label{identifikasi}

\begin{itemize}
\tightlist
\item
  Nama \& NIM penulis: \textbf{Armein Z. R. Langi -- 131902360} (tertera
  di halaman depan portofolio).
  (\href{https://ii-2100.github.io/all-about-me/}{II 2100})
\item
  Penilai: \textbf{Self-assessment (Armein Z. R. Langi)}
\item
  Catatan cakupan: halaman beranda memuat ``About Me''; navigasi ke ``My
  Songs for You'', ``My Stories for You'', ``My Shapes'', dan ``My
  Personal Reviews'' tersedia.
  (\href{https://ii-2100.github.io/all-about-me/}{II 2100})
\end{itemize}

\section{Tinjauan Umum}\label{tinjauan-umum}

\begin{itemize}
\tightlist
\item
  \textbf{UTS-1 (All About Me)} hadir di beranda (``Selamat Berjumpa /
  About Me''). Isi memperkenalkan identitas dan latar personal secara
  padat. (\href{https://ii-2100.github.io/all-about-me/}{II 2100})
\item
  \textbf{UTS-2 (My Songs for You)} memuat judul karya dan tautan audio,
  namun lirik/isi tidak ditampilkan di halaman (file audio tidak bisa
  saya akses dari sini), sehingga penilaian konten terbatas pada
  kelengkapan presentasi.
  (\href{https://ii-2100.github.io/all-about-me/My_Song_for_You/index.html}{II
  2100})
\item
  \textbf{UTS-3 (My Stories for You)} berisi tautan ke beberapa cerita;
  konten naratif kuat, terutama pada kisah keluarga ``Gaun Pengantin
  Gladys'' serta kisah inspiratif lainnya.
  (\href{https://ii-2100.github.io/all-about-me/My_Stories_for_You/index.html}{II
  2100})
\item
  \textbf{UTS-4 (My SHAPE)} saat ini halaman kosong/tanpa isi, hanya
  navigasi.
  (\href{https://ii-2100.github.io/all-about-me/My_Shapes/index.html}{II
  2100})
\item
  \textbf{UTS-5 (My Personal Reviews)} berisi metode/tautan panduan
  review, namun belum ada contoh \textbf{review personal} yang lengkap
  terhadap sebuah pesan/teks.
  (\href{https://ii-2100.github.io/all-about-me/My_Personal_Reviews/index.html}{II
  2100})
\end{itemize}

\begin{center}\rule{0.5\linewidth}{0.5pt}\end{center}

\section{Tinjauan Spesifik + Skor
(1--5)}\label{tinjauan-spesifik-skor-15}

\subsection{UTS-1 --- All About Me (di
beranda)}\label{uts-1-all-about-me-di-beranda}

\textbf{Skor per kriteria:} Orisinalitas \textbf{3}, Keterlibatan
\textbf{3}, Humor \textbf{1}, Wawasan/Insight \textbf{3} → \textbf{Total
10/20 (50\%)}. \textbf{Alasan singkat:} Perkenalan diri faktual dan
jelas, namun gaya masih informatif ketimbang naratif (minim humor dan
moment ``hook''). (\href{https://ii-2100.github.io/all-about-me/}{II
2100}) \textbf{Saran perbaikan:} Buka dengan anekdot pribadi (1--2
paragraf) yang ``mengikat'' (mis. titik balik karier/keluarga),
tambahkan satu momen humor ringan, lalu tutup dengan refleksi singkat
tentang nilai/visi diri agar aspek insight naik.

\subsection{UTS-2 --- My Songs for You}\label{uts-2-my-songs-for-you-1}

\textbf{Skor per kriteria:} Orisinalitas \textbf{2}, Keterlibatan
\textbf{2}, Humor \textbf{1}, Inspirasi \textbf{2} → \textbf{Total 7/20
(35\%)}. \textbf{Alasan singkat:} Halaman menampilkan judul lagu ``River
in my Mind'', ``Heaven on Earth'', namun tanpa lirik/cerita di balik
lagu sehingga sulit menilai aspek pesan, humor, dan inspirasi.
(\href{https://ii-2100.github.io/all-about-me/My_Song_for_You/index.html}{II
2100}) \textbf{Saran perbaikan:} Tambahkan lirik lengkap, 1 paragraf
cerita proses kreatif, dan 2--3 kalimat ``pesan untukmu'' agar inspirasi
terbaca; sertakan player/tautan yang dapat diputar langsung + fallback
transkrip.

\subsection{UTS-3 --- My Stories for
You}\label{uts-3-my-stories-for-you}

\textbf{Skor per kriteria:} Orisinalitas \textbf{5}, Keterlibatan
\textbf{5}, Pengembangan Narasi \textbf{4}, Inspirasi \textbf{5} →
\textbf{Total 19/20 (95\%)}. \textbf{Alasan singkat:} Cerita ``Gaun
Pengantin Gladys'' dkk sangat personal, emosional, dan inspiratif; ritme
bertutur hidup serta detail situasional kuat (konten ramu unsur
penebusan/keluarga/iman).
(\href{https://ii-2100.github.io/all-about-me/My_Stories_for_You/index.html}{II
2100}) \textbf{Saran perbaikan:} Tambah ``lead'' 2--3 kalimat yang
merangkum pesan kunci tiap cerita; akhiri dengan ajakan/refleksi 1--2
kalimat agar resonansi ke pembaca makin jelas.

\subsection{UTS-4 --- My SHAPE}\label{uts-4-my-shape-1}

\textbf{Skor per kriteria:} Orisinalitas \textbf{1}, Keterlibatan
\textbf{1}, Pengembangan \textbf{1}, Inspirasi \textbf{1} →
\textbf{Total 4/20 (20\%)}. \textbf{Alasan singkat:} Konten belum
tersedia.
(\href{https://ii-2100.github.io/all-about-me/My_Shapes/index.html}{II
2100}) \textbf{Saran perbaikan (prioritas):}

\begin{enumerate}
\def\labelenumi{\arabic{enumi}.}
\tightlist
\item
  Tampilkan rekap SHAPE (Spiritual gifts, Heart, Abilities, Personality,
  Experiences) dalam 1 tabel ringkas + 3--5 butir bukti;
\item
  Narasikan 2 pengalaman kunci (±150 kata/each) yang menunjukkan
  ``signature strengths'';
\item
  Tutup dengan 3 rencana aksi konkret (who/what/when).
\end{enumerate}

\subsection{UTS-5 --- My Personal
Reviews}\label{uts-5-my-personal-reviews-1}

\textbf{Skor per kriteria:} Pemahaman Konsep \textbf{2}, Analisis Kritis
\textbf{1}, Argumentasi (Logos) \textbf{1}, Etos \& Empati \textbf{2},
Rekomendasi \textbf{1} → \textbf{Total 7/25 (28\%)}. \textbf{Alasan
singkat:} Halaman berisi metode cara menilai, tetapi \textbf{belum ada}
contoh \textbf{review personal} yang lengkap terhadap sebuah pesan/teks
sehingga aspek analisis-argumentasi tak terbaca.
(\href{https://ii-2100.github.io/all-about-me/My_Personal_Reviews/index.html}{II
2100}) \textbf{Saran perbaikan:} Pilih 1 karya personal (mis.
UTS-1/2/3), tulis review 400--600 kata: ringkas pesan, nilai dengan
rubrik, berikan 2--3 bukti kutipan, evaluasi etos/empati, lalu tutup
dengan rekomendasi perbaikan spesifik.

\begin{center}\rule{0.5\linewidth}{0.5pt}\end{center}

\section{Rekap Skor (ringkas)}\label{rekap-skor-ringkas}

\begin{itemize}
\tightlist
\item
  \textbf{UTS-1:} 10/20 → \textbf{50\%}
\item
  \textbf{UTS-2:} 7/20 → \textbf{35\%}
\item
  \textbf{UTS-3:} 19/20 → \textbf{95\%}
\item
  \textbf{UTS-4:} 4/20 → \textbf{20\%}
\item
  \textbf{UTS-5:} 7/25 → \textbf{28\%}
\end{itemize}

CSV lengkap sudah saya siapkan untuk dokumentasi dan olah lanjut:
\href{sandbox:/mnt/data/UTS_self_assessment.csv}{Download CSV
ringkasan}.

\section{Langkah Perbaikan Cepat (prioritas 1
minggu)}\label{langkah-perbaikan-cepat-prioritas-1-minggu}

\begin{enumerate}
\def\labelenumi{\arabic{enumi}.}
\tightlist
\item
  \textbf{Lengkapi UTS-4 (My SHAPE)} sesuai saran di atas;
\item
  \textbf{Tambahkan lirik + sinopsis} untuk tiap lagu di UTS-2 agar bisa
  dinilai isi/inspirasinya;
\item
  \textbf{Tulis 1 review personal penuh} (UTS-5) memakai rubrik (pilih
  salah satu karya Anda sendiri sebagai objek);
\item
  Revisi UTS-1 dengan pembuka anekdot + refleksi nilai diri.
\end{enumerate}

Jika Anda mau, saya bisa bantu merapikan UTS-4 (tabel SHAPE + narasi)
dan membuat kerangka cepat untuk \textbf{review} di UTS-5 dari salah
satu karya Anda.

\bookmarksetup{startatroot}

\chapter{UAS-1 My Concepts}\label{uas-1-my-concepts}

Mau hidup epik ? \href{lifestory.pdf}{Write your Life Story}

Apa itu berkonsep?

\url{https://youtu.be/QVfUlVBO80U?si=yM6q_rwV9rcDBbu7}

\bookmarksetup{startatroot}

\chapter{UAS-3 My Opinions}\label{uas-3-my-opinions}

SApa itu beropini? \href{BM\%20Opini.mp4}{Opini Berpengaruh}

Bagiamana menjaadi menarik? \href{./Interesting.mp4}{Menjadi Menarik}

\bookmarksetup{startatroot}

\chapter{UAS-3 My Innovations}\label{uas-3-my-innovations}

\bookmarksetup{startatroot}

\chapter{UAS-4 My Knowledge}\label{uas-4-my-knowledge}

Cara saya mengkomunikasikan sebuah pengatahuan sebagai petunjuk bagi
orang lain 1) saya tulis
\href{Rekomendasi\%20Presentasi\%20Efektif(Contoh\%20Makalah).pdf}{makalah
sebagai bahan utama} 2) lalu saya buat
\href{Contoh\%20Transkrip\%20Presentasi.pdf}{transkrip ucapan lisan} 3)
kemudian saya kembangkan
\href{Rekomendasi\%20Presentasi\%20(Contoh\%20Slides).pdf}{slide
pendukung trnsskrip} 4) lalu saya memproduksivideo audio visual
\url{https://youtu.be/ZbghfMvnPZc} \url{https://youtu.be/ZbghfMvnPZc}

\bookmarksetup{startatroot}

\chapter{UAS-5 My Professional
Reviews}\label{uas-5-my-professional-reviews}

Untuk melAkukan review, seperti pada
\href{../My_Personal_Reviews/Doc.5.Mengevaluasi-Esai-Berdasarkan-Rubrik.pdf}{pendekatan
AI}, kita membutuhkan rubrik

\bookmarksetup{startatroot}

\chapter{Summary}\label{summary}

In summary, this book has no content whatsoever.

\bookmarksetup{startatroot}

\chapter*{References}\label{references}
\addcontentsline{toc}{chapter}{References}

\markboth{References}{References}

\phantomsection\label{refs}




\end{document}
